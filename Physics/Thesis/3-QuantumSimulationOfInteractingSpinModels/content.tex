\section{Quantum Simulation Of Interacting Spin Models With Trapped Ions} % (fold)
\label{sec:quantum_simulation_of_interacting_spin_models_with_trapped_ions}

\subsection{Introduction} % (fold)
\label{sub:introduction}


\begin{itemize}
\item
Feynman`s idea could be addressed as a digital quantum simulator, and the Hamiltonian is constructed from piecewise application of local Hamiltonians, Trotter expansion
\begin{equation}\label{eq:trotter}
    e^{-iHt}\approx \qty(e^{-iH_1t/n}e^{-iH_2t/n}e^{-iH_3t/n}\cdots e^{-iH_lt/n})^n
\end{equation}
Thus, error in simulating Hamiltonian can be kept uder a given value by property choosing the number of steps.

\item
Analog quantum simulator follows the mathematically equivalent evolution, restricted into a few classes of Hamiltonians.
\begin{itemize}
    \item strongly correlated system
    \item high temperature superconductors
    \item heavy fermion materials
    \item quantum Monte Carlo
    \item density matrix renormalization group
\end{itemize}
\end{itemize}

\subsection{Trap Setup} % (fold)
\label{sub:trap_setup}
The trap is a three layer linear Paul trap with $6$ DC electrodes and $6$ ground electrodes and $2$ RF electrodes. The bottom and top layers of electrodes are approximately $250\unit{\mu m}$ and the middle layer of RF electrodes are $125\unit{\mu m}$
% subsection trap_setup (end)
\begin{itemize}
    \item RF frequency $38.6\unit{MHz}$, and the quality factor is about $200$. The input power to the helical resonator is approximately $27\unit{dBm}$($500\unit{mW}$), which may generate a radio-frequency voltage of about $200--300$ volts, leading to secular frequencies of $\omega_x\approx\omega_y\approx 2\pi \times 5\unit{MHz}$
    \item
    \[V_{end}=\frac{V_1+V_2+V_5+V_6}{4},\quad V_{central}=\frac{V_3+V_4}{2}\]
    And the voltage is generated by High precision HV module from ISEG. The central and end voltages can be changed to manipulate the principla axes of the trap along the transverse directions.
    \item
    The Z-push voltage $V_z=\frac{V_1+V_5-V_2-V_6}{2}$ controls the ions position along Z-axis.
    \item The end difference and central difference are used to minimize the radio frequency micromotion
\end{itemize}


\subsection{Ion Trapping} % (fold)
\label{sub:ion_trapping}
Static electric field is impossible to create a 3-D stationary point for charged particle, thus we need to use some radio frequency fields for ion trapping theory with an effective confining potential, which can be demonstrated by only one dimension
\begin{equation}
    E(x)=E_0(X)\cos\Omega t
\end{equation}

If the field is homogeneous, then $E_0(X)$ is a constant with varying $X$, then the charged particle could be described as a dynamic equation
\begin{equation}\label{eq:dynamics}
    m\ddot{X}(t)=F(t)=eE_0(X_0)\cos\Omega t
\end{equation}
The solution can be easily obtained as
\begin{equation}
    X(t)=-\frac{eE_0}{m\Omega_{rf}^2}\cos\Omega_{rf}t+X_0
\end{equation}
Now, we have already known the solution to a homogeneous electric field for charged particles. Furthermore, we can approximate the solution to a electric field with a small inhomogeneity
\begin{equation}
    E_0(X)=E_0(X_0)+\pdv{E_0(X)}{X}\eval{}_{X=X_0}\qty(X-X_0)
\end{equation}
Where we suppose the displacement is small enough so that we can approximate the solution with above homogeneous electric field.

The force applied on the charged particle can be now rewritten as
\begin{equation}
\begin{aligned}
    F_X(t)&=e(E_0(X_0)+\pdv{E_0(X)}{X}\eval{}_{X=X_0}\qty(X-X_0))\cos\Omega t\\
    &=eE_0(X_0)\cos\Omega t+e\pdv{E_0(X)}{X}\eval{}_{X=X_0}\qty(X-X_0)\cos\Omega t\\
    &=eE_0(X_0)\cos\Omega t-\frac{e^2E_0(X_0)}{m\Omega_{rf}^2}\pdv{E_0(X)}{X}\eval{}_{X=X_0}\cos^2\Omega t
\end{aligned}
\end{equation}
The time average of this force is

\begin{equation}\label{eq:timeaveraged}
\begin{aligned}
    F_X(t)&=-\frac{e^2E_0(X_0)}{2m\Omega_{rf}^2}\pdv{E_0(X)}{X}\eval{}_{X=X_0}\\
    &=-e\pdv{X}(\frac{eE_0^2(X_0)}{4m\Omega_{rf}^2})\eval{}_{X=X_0}
\end{aligned}
\end{equation}
Thus we can define a effective potential which can be denoted by ponderomotive potential for time averaged confining potential and the ponderomotive is independent of the electric charge sign(negative or positive)
\begin{equation}\label{eq:ponderomotive}
    \Psi_{\mathrm{pond}}(X)=\frac{eE_0^2(X)}{4m\Omega_{rf}^2}
\end{equation}
In addition, the region of $E_0(X)=0$ is referred to as a \emph{radio-frequency null}, which can be a point, a collection of discrete points or a line depending on the geometry of the trap. And the static potentials are adjusted so that the micromotion can be minimized, which can avoid the coupling to vibrational modes of the ion chain and resulting in quantum decoherence by heating up the modes.

\subsection{Ion Loading} % (fold)
\label{sub:ion_loading}
The photoionization of the neutral Yb is a two-photon ionization which includes the $^1S_)$ level to the $^1P_1$ level, and then $^1P_1$ continuum or more. As a typical setup, $1\unit{mW} 399$ laser with beam waist about $100\unit{\mu m}$ and the second step can be performed by any light below $394.1\unit{nm}$. Typically, we can use $369.5\unit{nm}$ as a ionizaiton light or $355\unit{nm}$ with a typical energy $1\unit{mW}$, but single ions can be loaded one by one with $369.5$, however multiple ions will be loaded simutaneously with $355$.

The other isotopes will be a dark spot in the $^{171}Yb$ion chain, the isotope shift between the $^{171}Yb$ and $^{174}Yb$ in the $^1S_0-^1P_1$ transition frequency is about \emph{$800\unit{MHz}$} which is more than the broadened width($200\unit{MHz}$ typically)
The protection beam which is red detuned from the $^2S_{1/2}\rightarrow ^2P_{1/2}$ resonance by $600\unit{MHz}$, and it is on during the loading.
