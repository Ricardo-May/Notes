\section{Quantum Simulation Of Interacting Spin Models With Trapped Ions} % (fold)
\label{sec:quantum_simulation_of_interacting_spin_models_with_trapped_ions}

\subsection{Introduction} % (fold)
\label{sub:introduction}


\begin{itemize}
\item
Feynman`s idea could be addressed as a digital quantum simulator, and the Hamiltonian is constructed from piecewise application of local Hamiltonians, Trotter expansion
\begin{equation}\label{eq:trotter}
    e^{-iHt}\approx \qty(e^{-iH_1t/n}e^{-iH_2t/n}e^{-iH_3t/n}\cdots e^{-iH_lt/n})^n
\end{equation}
Thus, error in simulating Hamiltonian can be kept uder a given value by property choosing the number of steps.

\item
Analog quantum simulator follows the mathematically equivalent evolution, restricted into a few classes of Hamiltonians.
\begin{itemize}
    \item strongly correlated system
    \item high temperature superconductors
    \item heavy fermion materials
    \item quantum Monte Carlo
    \item density matrix renormalization group
\end{itemize}
\end{itemize}

\subsection{Trap Setup} % (fold)
\label{sub:trap_setup}
The trap is a three layer linear Paul trap with $6$ DC electrodes and $6$ ground electrodes and $2$ RF electrodes. The bottom and top layers of electrodes are approximately $250\unit{\mu m}$ and the middle layer of RF electrodes are $125\unit{\mu m}$
% subsection trap_setup (end)
\begin{itemize}
    \item RF frequency $38.6\unit{MHz}$, and the quality factor is about $200$. The input power to the helical resonator is approximately $27\unit{dBm}$($500\unit{mW}$), which may generate a radio-frequency voltage of about $200--300$ volts, leading to secular frequencies of $\omega_x\approx\omega_y\approx 2\pi \times 5\unit{MHz}$
    \item
    \[V_{end}=\frac{V_1+V_2+V_5+V_6}{4},\quad V_{central}=\frac{V_3+V_4}{2}\]
    And the voltage is generated by High precision HV module from ISEG. The central and end voltages can be changed to manipulate the principla axes of the trap along the transverse directions.
    \item
    The Z-push voltage $V_z=\frac{V_1+V_5-V_2-V_6}{2}$ controls the ions position along Z-axis.
    \item The end difference and central difference are used to minimize the radio frequency micromotion
\end{itemize}


\subsection{Ion Trapping} % (fold)
\label{sub:ion_trapping}
Static electric field is impossible to create a 3-D stationary point for charged particle, thus we need to use some radio frequency fields for ion trapping theory with an effective confining potential, which can be demonstrated by only one dimension
\begin{equation}
    E(x)=E_0(X)\cos\Omega t
\end{equation}

If the field is homogeneous, then $E_0(X)$ is a constant with varying $X$, then the charged particle could be described as a dynamic equation
\begin{equation}\label{eq:dynamics}
    m\ddot{X}(t)=F(t)=eE_0(X_0)\cos\Omega t
\end{equation}
The solution can be easily obtained as
\begin{equation}
    X(t)=-\frac{eE_0}{m\Omega_{rf}^2}\cos\Omega_{rf}t+X_0
\end{equation}
Now, we have already known the solution to a homogeneous electric field for charged particles. Furthermore, we can approximate the solution to a electric field with a small inhomogeneity
\begin{equation}
    E_0(X)=E_0(X_0)+\pdv{E_0(X)}{X}\eval{}_{X=X_0}\qty(X-X_0)
\end{equation}
Where we suppose the displacement is small enough so that we can approximate the solution with above homogeneous electric field.

The force applied on the charged particle can be now rewritten as
\begin{equation}
\begin{aligned}
    F_X(t)&=e(E_0(X_0)+\pdv{E_0(X)}{X}\eval{}_{X=X_0}\qty(X-X_0))\cos\Omega t\\
    &=eE_0(X_0)\cos\Omega t+e\pdv{E_0(X)}{X}\eval{}_{X=X_0}\qty(X-X_0)\cos\Omega t\\
    &=eE_0(X_0)\cos\Omega t-\frac{e^2E_0(X_0)}{m\Omega_{rf}^2}\pdv{E_0(X)}{X}\eval{}_{X=X_0}\cos^2\Omega t
\end{aligned}
\end{equation}
The time average of this force is

\begin{equation}\label{eq:timeaveraged}
\begin{aligned}
    F_X(t)&=-\frac{e^2E_0(X_0)}{2m\Omega_{rf}^2}\pdv{E_0(X)}{X}\eval{}_{X=X_0}\\
    &=-e\pdv{X}(\frac{eE_0^2(X_0)}{4m\Omega_{rf}^2})\eval{}_{X=X_0}
\end{aligned}
\end{equation}
Thus we can define a effective potential which can be denoted by ponderomotive potential for time averaged confining potential and the ponderomotive is independent of the electric charge sign(negative or positive)
\begin{equation}\label{eq:ponderomotive}
    \Psi_{\mathrm{pond}}(X)=\frac{eE_0^2(X)}{4m\Omega_{rf}^2}
\end{equation}
In addition, the region of $E_0(X)=0$ is referred to as a \emph{radio-frequency null}, which can be a point, a collection of discrete points or a line depending on the geometry of the trap. And the static potentials are adjusted so that the micromotion can be minimized, which can avoid the coupling to vibrational modes of the ion chain and resulting in quantum decoherence by heating up the modes.

\subsection{Ion Loading} % (fold)
\label{sub:ion_loading}
The photoionization of the neutral Yb is a two-photon ionization which includes the $^1S_)$ level to the $^1P_1$ level, and then $^1P_1$ continuum or more. As a typical setup, $1\unit{mW}  399$ laser with beam waist about $100\unit{\mu m}$ and the second step can be performed by any light below $394.1\unit{nm}$. Typically, we can use $369.5\unit{nm}$ as a ionizaiton light or $355\unit{nm}$ with a typical energy $1\unit{mW}$, but single ions can be loaded one by one with $369.5$, however multiple ions will be loaded simutaneously with $355$.

The other isotopes will be a dark spot in the \yb{171} ion chain, the isotope shift between the \yb{171} and \yb{174} in the $^1S_0-^1P_1$ transition frequency is about \emph{$800\unit{MHz}$} which is more than the broadened width($200\unit{MHz}$ typically)
The protection beam which is red detuned from the $^2S_{1/2}\rightarrow ^2P_{1/2}$ resonance by $600\unit{MHz}$, and it is on during the loading.
\subsection{Crystal Recapture} % (fold)
\label{sub:crystal_recapture}
The collisions with background gases(mostly Hydrogen) will cause the melting of ion crystal, the probability will increase with increasing ion chain number. Typically, the collision event will occur per five minutes on an average for a chain of $10$ ions. In order to recapture the crystal, Doppler cooling beam and protection beam should be turned on and the trap depth should be lowered by $11\unit{dB}$ for its RF power, and the DC depth should be also reduced to a lower level.

\subsection{\yb{171} qubit} % (fold)
\yb{171} is a really nice qubit candidate with magnetically insensitive two-level system
, and the nucleus has a spin $1/2$ so that there exist a hyperfine structure in the electronic levels. The ground state is $^2S_{1/2}$, and the hyperfine splitting cause it into two states $^2S_{1/2}\ket{F=1,m_F=0}$, $^2S_{1/2}\ket{F=0,m_F=0}$, and the $F=1$ states has three manifolds with projection into $m_F=-1, m_F=0, m_F=1$, which can be splitted by a externally applied magnetic field $B_Y\approx 5\unit{G}$ with a zeeman splitting $1.4\unit{MHz/G}\times B_Y$, however the hyperfine splitting is a second-order Zeeman splitting which could be denoted by
\begin{equation}\label{eq:hyperfinesplitting}
    \delta_{zz}=\qty(310.8)B^2\unit{Hz},\quad B \textrm{ in Gauss}
\end{equation}
and the hyperfine splitting without external magnetic field is $12642.812118466\unit{MHz}$. However, the \yb{174} has no nucleus spin so that there is no hyperfine structure in its electronic levels.

Doppler cooling can be considered as a momentum kick with different directions which occupy different probabilities, thus there are more atoms absorb photons moving in opposite direction with a red detuned $369.52\unit{nm}$ laser beam from resonance about $25\unit{MHz}$, the transition occurs on $^2S_{1/2}--^2P_{1/2}$. Furthermore, the Doppler cooling is exerted on the hyperfine states that a sideband is required($1\%$ of the carrier strength) that couples the $^2S_{1/2}\ket{F=0,m_F=0}$ and $^2P_{1/2}\ket{F=1,m_F=0}$, and the frequency difference is $14.74\unit{GHz}$. The optical power in the cooling beam used is approximately $25\unit{\mu W}$ focused to a spot $100\unit{\mu m}\times 10\unit{\mu m}$ for about $3\unit{ms}$. Furthermore, a re-pump laser at $935.2\unit{nm}$ and the sideband at $3.07\unit{GHz}$ is generated by fiber-eom that re-pump $^2D_{3/2}$ that $^2P_{1/2}$ states leak into with probability of $0.005$ upto $^3D[3/2]_{1/2}$that will decays back to the cooling cycle $^2S_{1/2}$ without mixing the qubit states, because the transition $^3D[3/2]_{1/2}\ket{F=0,m_F=0}\rightarrow^2S_{1/2}\ket{F=0,m_F=0}$ is forbidden. And the laser beam power is about $20\unit{mW}$ and is not focused tightly(about hundreds of microns beam waist at the ions), the the laser is frequency stabilized using the software lock(digital PID). Finally, there is also a leak into the $10$-year state $^2F_{7/2}$, then the $638.6\unit{nm}$ laser is scanned from $38.6102$ to $638.6151$ to repump it back to the main cooling cycle, and the event occurs once every couple of hours for s single atom.

Detection of qubit states is implemented by state-dependent fluorescence, $^2S_{1/2}\ket{F=1,m_F=0}\leftrightarrow ^2P_{1/2}\ket{F=0,m_F=0}$, then the ion decays into one of three manifolds of $^2S_{1/2}\ket{F=1}$ after about $10\unit{ns}$, in order to repump all Zeeman states of $^2S_{1/2}$ manifold, all the three polarizations is required and an external magnetic field of about $5\unit{G}$ is also needed to avoid coherent population trapping, and a typical duration of detection is $800\unit{\mu s}$, and the fidelity of detection is dependent of detection devices, PMT($98.5\%$) and CCD($93\%$).

One of the basic task to implement universal quantum computing is initialization of qubit state, with \yb{171} we can use optical pumping to intialize the qubit into \dark state. $369.53$ beam is frequency modulated on resonance with $\bright\leftrightarrow ^2P_{1/2}\ket{F=0,m_F=0}$ by $2.105\unit{GHz}$ using Qubig EOM. Then the $^2P_{1/2}\ket{F=1}$ manifolds will decay into \dark state. And the \dark state is off-resonant from $^2P_{1/2}$ states by $12.642812\unit{GHz}$ so that scatter hardly occurs, this setup will complete a perfect initialization more than $99\%$ within $3\unit{\mu s}$.

\subsection{Raman manipulation of qubits} % (fold)
\label{sub:raman_manipulation_of_qubits}
The single-qubit quantum gate or operation could be realized by microwave or two photon laser induced stimulated Raman transition. Microwave at $12.642812\unit{GHz}$ could be used to rotate the two qubit states, however it does not have sufficient momentum to excite the vibrational modes so that the quantum simulation mediated by phonon modes could be performed by microwave techniques. Thus stimulated Raman transition is a better choice for quantum simulation.

The frequency detuning should be much larger than the Rabi frequency $g$ of the system so that the system could be approximated as a two level system with effective Rabi frequency $\Omega=g^*g/2\Delta$
